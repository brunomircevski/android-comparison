\section{Introduction}
Android is the dominant mobile operating system and acts as a primary gateway to digital world for billions of users. It's crucial for core OS to be a reliable, stable and trustworthy platform, that allows users to run apps and services of their choice. While mobile app privacy has been widely studied \cite{appDataCollection1,appDataCollection2,appDataCollection3,appDataCollection4}, less attention is typically paid to the operating system layer and to privileged service frameworks that silently enable additional functionality. Standard unprivileged apps are constrained by Android's permission model, although some attempt to circumvent it in creative ways \cite{appDataCollection5Circumvention,appDataCollection6Circumvention}. In contrast, privileged system apps operate with higher access to the device and OS, which makes them harder to understand and control. Most user-facing privacy controls focus on applications and permissions, yet a substantial amount of data exchange can occur below the app layer, through OS services, vendor components, and Google service frameworks. This thesis therefore compares the network behavior of multiple Android distributions and their preinstalled/privileged system components under controlled, reproducible conditions.

\subsection{Android dependency on privacy-threatening services}
While Android’s core is the Android Open Source Project (AOSP), most consumer devices ship with additional proprietary components layered on top. They are often implemented as privileged system apps and service frameworks to provide baseline functionality such as push notifications, location assistance, app distribution/updates, device integrity checks and many more. Users and app developers commonly assume their presence and availability by default. 

These components exists for a practical reason. Centralized services can significantly improve usability and efficiency, for example by reducing battery drain compared to per‑app background connections, and by providing faster, more accurate location through fused network and sensor-based methods. However, this design also tightly couples many functions into a single service stack, so enabling useful capabilities may implicitly enable others that a user would not choose, such as advertising identifiers, telemetry, and behavioral tracking. In practice, the result is a large proprietary bundle that is difficult to audit and decompose into only the features a user actually wants.

On Android, these baseline capabilities are most commonly delivered through Google’s service stack. Because Google’s business model is strongly advertising-driven, many users may be unwilling to entrust this provider with broad access to device and their data. In principle, users should be able to choose which services they rely on and have confidence that these choices are enforced by the platform’s permission and isolation mechanisms. This demand for transparency and controllability motivates alternative approaches such as microG and GrapheneOS’ sandboxed Google Play. These solutions aim to preserve device functionality and app compatibility while reducing privilege and improving user control over Google-related components.

\subsection{Understanding the threat model}
\subsection{Privacy-focused Android operating systems}
\subsection{Challenges of network traffic analysis on Android}