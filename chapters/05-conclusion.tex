\section{Discussion}

Our measurements reveal a clear hierarchy in how different Android configurations handle user data and privacy. The most fundamental finding is that adjusting privacy settings on stock Android, while beneficial, cannot achieve the same protection as switching to an alternative distribution. On system running proprietary Google services, privacy adjusted scenarios showed modest data transmission reductions compared to default setup. Their underlying architecture tightly integrates Google Mobile Services into core OS and persists background activity. Substantial data flows to Google infrastructure continued regardless of user-facing privacy toggles.

Alternative distributions shift the privacy posture by making Google communication user-initiated rather than ambient. All privacy-focused systems generated orders of magnitude less network traffic with Google. Some of them are utilizing their own or third-party privacy-focused infrastructure to replace Google's. LineageOS with GMS was an exception, while more minimal, by having official Google services preinstalled, it does not differ much from stock Android in terms of privacy. Among other configurations enabling basic Google functionality, microG represents the most privacy-preserving approach, though this comes with potential security trade-offs (e.g. signature spoofing). Our measurements show microG transmits the least data to Google infrastructure while maintaining basic service compatibility. It achieves this through identifier randomization rather than exposing authentic hardware identifiers, similar to GrapheneOS. Our observations illustrate the fundamental difference: microG optimizes for privacy, while GrapheneOS prioritizes platform security, while still delivering great privacy, substantially improved over stock systems.

Alternative distributions also improve user control through minimal base systems and enhanced permission granularity. All tested alternatives ship without forced user agreements during initial setup, unlike stock Android's mandatory terms requiring data collection. One privacy-friendly feature we found in all cases is network permission controls that stock lacks, allowing users to deny Internet access to specific apps. Aurora Store demonstrates that app sourcing from official repositories remains viable without direct Google infrastructure interaction, retrieving identical packages from the same CDN endpoints while generating fewer requests and adding privacy-focused metadata lookups. Updates on alternative systems are typically smaller, more transparent, and user-controlled rather than automatic and opaque.

Several limitations qualify these findings. Our measurements capture specific time-bounded scenarios rather than universal behavior. Traffic patterns likely vary with extended use and evolving service configurations. The TLS interception approach, while revealing substantial information, remains incomplete due to certificate pinning, anti-tamper mechanisms, and protocol evolution. GrapheneOS's resistance to our instrumentation, while security-advantageous, means we cannot closely verify what data sandboxed services transmit. These constraints mean our findings indicate general trends and architectural differences rather than definitive absolute measurements.

Based on the network traffic we analyzed, Google can potentially distinguish between stock Android and alternative distributions through device characteristics transmitted during service interactions, raising the possibility of selective enforcement or access restrictions. Ultimately, the choice between privacy-focused alternative operating systems depends on individual threat models and functional requirements, but all represent meaningful advances in user control and privacy over stock Android.

\section{Conclusion}

In this work we compared network behavior across multiple Android distributions, measuring data transmission during setup, idle periods, and basic app use. We evaluated Stock Android, LineageOS variants, iodéOS, and GrapheneOS to quantify how different approaches to Google services affect privacy. Our findings show that alternative distributions reduce unsolicited data flows by orders of magnitude compared to stock systems. These systems improve privacy by reducing data exposure, limiting or randomizing identifiers, and using independent infrastructure. TLS interception and Google Takeout analysis revealed that stock Android transmits persistent hardware identifiers, app inventories, and telemetry, where alternatives do not or limit such transmission substantially.

Future work could extend measurements to longer usage periods and repeated trials to distinguish systematic differences from transient variation. More realistic tasks incorporating diverse apps and connectivity changes would better approximate everyday use. As Android, Google services and alternatives evolve, repeating this study would reveal whether subsequent privacy improvements meaningfully change the patterns we observed.