\section{Related work}
Prior work has established that Android privacy risks arise not only from user-installed applications but also, in some cases, from the operating system and its preinstalled components. Privacy-focused Android distributions have gained popularity as alternatives to stock vendor firmware, yet they remain comparatively understudied in the measurement literature. Consequently, there is limited empirical understanding of what practical privacy and control benefits different Android distributions provide to users.

A key OS-level measurement study addresses this gap by comparing several vendor-customized Android variants with two open-source distributions under a privacy-conscious baseline configuration \cite{android-os-snooping2021}. It reports that all tested stock/vendor builds transmit substantial data to OS vendors and third parties even when the handset is idle, and that this collection persists without an effective opt-out. In contrast, the privacy-focused open-source /e/OS in their comparison exhibits minimal data transmission, whereas LineageOS still shows substantial communication to Google because it was evaluated with Google’s proprietary app stack present. This distinction is important for studies like ours that evaluate multiple distributions and configurations (including variants with and without integrated Google services), since the presence of Google system components can dominate observed traffic and shape the overall privacy profile.

Follow-up work focusing specifically on OEM telemetry further shows that Samsung, Xiaomi, Huawei and Realme make extensive use of long-lived hardware identifiers and also collect installed-app lists and analytics data, sometimes even in connections that should not require persistent identifiers, enabling straightforward linkage to a user’s identity when an OEM account is used \cite{android-os-snooping2023}.

Additional work by the same authors analyzes telemetry from two core Google system apps, Google Messages and Google Dialer, by decrypting and decoding the logging channels that forward data to Google \cite{google-dialer-messages}. The study reports that these apps disclose sensitive communication metadata, including when SMS messages and calls are sent or received, with timestamps and for calls duration. Messages also sends a hash derived from message content that can uniquely identify messages and link participants. It further finds that phone numbers may be transmitted and that events are tagged with persistent identifiers such as the Android ID, undermining anonymity and leaving users with no effective opt out. Taken together, these results suggest that even basic default communication apps on Google Android cannot be assumed to be trustworthy from a privacy perspective.

Complementing OS-level telemetry measurements, a recent study focuses specifically on what pre-installed Google components persistently store on-device and shows that Google’s role in Android privacy extends beyond outbound traffic \cite{cookies-identifiers2025}. It reports that Google Play Services, the Play Store, and other bundled Google apps receive and store multiple cookies, advertising-related identifiers, and tracking links even after a factory reset and even when the user has not opened Google apps, with no dedicated consent prompt and no practical opt-out. The study highlights the Google Android ID as a persistent identifier provisioned early and widely reused in subsequent Google connections, and documents how additional tokens and cookies can be stored and later transmitted alongside telemetry or analytics events. Together, these findings reinforce that Google’s integrated service stack functions as a central privacy-relevant layer of the Android ecosystem, shaping device identification and data handling independently of user-installed applications.

A broader view of Google’s privacy impact is given in a report that describes data collection as an ecosystem across platforms, apps, and advertising services, not as a single feature of Android \cite{google-data-collection2018}. It distinguishes "active" data sharing (when users directly use Google products) from "passive" collection that happens in the background via Android/Chrome and via tracking/analytics components embedded in many third-party apps and websites. The report argues that these different data sources can be combined to build detailed user profiles, and that "pseudonymous" identifiers (like ad IDs or cookies) can still be linked back to user accounts when they appear together in the same workflows.

Overall, prior work consistently indicates that Google’s integrated service stack is a dominant driver of privacy-relevant data flow on Android. It enables persistent device identification, background telemetry, and cross-context linkage between app usage, web activity, and advertising infrastructure, often with limited user visibility and weak practical opt-out mechanisms. At the same time, the literature suggests that meaningful reductions in unsolicited communication are possible when Google components are absent or replaced with open source variants \cite{android-os-snooping2021}. Alternative configurations and distributions remain less widely deployed, less familiar to typical users, and comparatively under-explored in reproducible measurement studies. This motivates systematic, real-device comparisons of stock systems versus privacy-oriented alternatives and configurations to quantify what privacy improvements are achievable in practice and what trade-offs they entail.