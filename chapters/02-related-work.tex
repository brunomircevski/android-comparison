\section{Related work}
Prior work has shown that privacy risks on Android extend beyond user-installed applications and can arise from the operating system layer and preinstalled components. Measurement studies of vendor-modified Android variants report background communication to OS vendors and third parties even under minimal configuration. Transmitted data includes persistent identifiers, app and device metadata, which creates possibility of re-linking resettable identifiers when they co-occur with long-lived identifiers \cite{android-os-snooping2021}. Other measurements report that Google Play Services and the Play Store store cookies, tokens, and device identifiers even in the absence of explicit user activity \cite{cookies-identifiers2025}. Broader ecosystem-level investigations further highlight how data collection can be driven not only by first-party apps and platforms, but also by advertising and analytics infrastructure (which is sometimes preinstalled), enabling linkage between nominally pseudonymous identifiers and user accounts \cite{google-data-collection2018}.

[Alternative OS are not well studied...]